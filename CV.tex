% arara: pdflatex
% arara: pdflatex
% arara: clean: { files: [CV.aux, CV.out, CV.toc, CV.log]}

\documentclass[11pt,letterpaper,sans]{moderncv}
% moderncv themes
\moderncvtheme[blue]{casual}
%\usepackage[spanish,es-lcroman]{babel}
\usepackage[utf8x]{inputenc}
%\usepackage{hyperref}
\usepackage[scale=0.8]{geometry}
\usepackage{amsfonts}
\usepackage{amsmath}
\usepackage{amssymb}
\usepackage{anysize}
\usepackage{xcolor}

%\DefineNamedColor{named}{RoyalPurple}{cmyk}{0.75,0.90,0,0}
\definecolor{RoyalBlue}{cmyk}{1,0.50,0,0}

\newcommand{\superscript}[1]{\ensuremath{^{\textrm{#1}}}}
\newcommand{\subscript}[1]{\ensuremath{_{\textrm{#1}}}}
%\setlength{\hintscolumnwidth}{3cm}% if you want to change the width of the column with the dates
%\AtBeginDocument{\setlength{\maketitlenamewidth}{6cm}}% only for the classic theme, if you want to change the width of your name placeholder (to leave more space for your address details
%\AtBeginDocument{\recomputelengths}% required when changes are made to page layout lengths

% Personal Data
\firstname{Jaime Jesús}
\familyname{\\Delgado Meraz}
\title{Master en Sistemas Computacionales}
\address{Calle Río Eufrates \#151}{Fracc. Praderas del Río, Ciudad Valles, S.L.P.}
\mobile{4811155821}
\phone{(481)3815957}
\email{j2deme@gmail.com}
%\homepage{homepage (optional)}                % optional, remove the line if not wanted
%\extrainfo{additional information (optional)} % optional, remove the line if not wanted
\photo[64pt]{IMG_0847}                         % '64pt' is the height the picture must be resized to and 'picture' is the name of the picture file; optional, remove the line if not wanted
%\quote{Some quote (optional)}                 % optional, remove the line if not wanted

% to show numerical labels in the bibliography; only useful if you make citations in your resume
\makeatletter
\renewcommand*{\bibliographyitemlabel}{\@biblabel{\arabic{enumiv}}}
\makeatother

% bibliography with mutiple entries
%\usepackage{multibib}
%\newcites{book,misc}{{Books},{Others}}

%\nopagenumbers{}% uncomment to suppress automatic page numbering for CVs longer than one page
%----------------------------------------------------------------------------------
%            content
%----------------------------------------------------------------------------------
\begin{document}
\maketitle

\section{Información Personal}
\cvline{Fecha de Nacimiento}{21 Agosto 1987}
\cvline{Lugar de Nacimiento}{Ciudad Valles, San Luis Potosí. México}
\cvline{Nacionalidad}{Mexicana}
\cvline{Estado Civil}{Casado}
\cvline{CURP}{DEMJ870821HSPLRM03}
\cvline{RFC}{DEMJ870821AA3}
\cvline{Cédula Licenciatura}{6495229}
\cvline{Cédula Maestría}{\emph{En trámite}}

\section{Tesis Maestría}
\cvline{Título}{\emph{Plataforma web para el análisis de la gesticulación facial.}}
\cvline{Asesor}{Dr. Pedro Luis Sánchez Orellana}

\section{Residencia Profesional Licenciatura}
\cvline{Título}{\emph{AdaptText: Middleware basado en perfiles para la administración, supervisión y generación dinámica de contenidos de texto.}}
\cvline{Asesor}{Dr. José Luis González Compeán}

\section{Formación Académica}
\cventry{2011 -- 2013}{Maestría Profesionalizante en Sistemas Computacionales}{Instituto Tecnológico de Ciudad Victoria}{Cd. Victoria, Tamaulipas}{}{Especialidad en Ingeniería de Software}
\cventry{2005 -- 2009}{Licenciado en Informática}{Instituto Tecnológico de Cd. Valles}{Cd. Valles, San Luis Potosí}{}{Especialidad en Gestión de Tecnologías de Información}
\cventry{2002 -- 2005}{Especialidad en Informática}{Colegio de Bachilleres Plantel 06}{Cd. Valles, San Luis Potosí}{}{}

\section{Experiencia Docente}
\cventry{Enero 2014 a la fecha}{Profesor de Inglés}{Instituto Tecnológico de Ciudad Valles}{Cd. Valles, S.L.P}{}{Profesor en cursos de Inglés para alumnos del Instituto Tecnológico de Ciudad Valles}
\cventry{Enero 2013 a la fecha}{Profesor}{Instituto Tecnológico de Ciudad Valles}{Cd. Valles, S.L.P}{}{Profesor en asignaturas de las carreras de Ingeniería en Tecnologías de la Información e Ingeniería en Sistemas Computacionales}
\cventry{Agosto 2011 -- Diciembre 2012}{Profesor de Inglés}{Universidad Politécnica de Victoria}{Cd. Victoria, Tamps}{}{Profesor en cursos de Inglés para las carreras de Ingeniería en Tecnologías de la Información e Ingeniería en Mecatrónica.}
\cventry{Enero 2008 -- Julio 2010}{Profesor de Inglés}{Centro de Idiomas Universitario, Universidad Autónoma de San Luis Potosí Zona Huasteca}{Cd. Valles, S.L.P}{}{Profesor en cursos de Inglés de 40, 48 y 80 horas a ni\~nos, adolescentes y adultos.}

\section{Experiencia en Investigación}
\cvline{Octubre 2015}{\textbf{Colaborador} activo del cuerpo académico \emph{Aplicación de Sistemas para el Manejo de Grandes Volúmenes de Datos de Ubicación Geográfica, Científica y Lingüística}.}

\subsection{Habilidades de Investigación}
\cvline{}{Conocimiento de sistemas de discos paralelos, distribución de bloques en sistemas de almacenamiento para sistemas distribuidos, simuladores de arquitecturas de almacenamiento, construcción de módulos para sistemas de distribución de bloques en sistemas RAID y algoritmos de cómputo lingüístico para la clasificación, organización y visualización de acervo digital.}

\subsection{Publicaciones}
\cvline{2015}{\textbf{NENEK-EGAD: Esquema de Almacenamiento para la gestión del acervo digitalizado.} R. M. Jiménez Maldonado, D. R. Hernández López, J. J. Delgado Meraz. XLII Conferencia Nacional de Ingeniería de la ANFEI. Ensenada, Baja California, México, Junio 15 -- 17, 2015.}
\cvline{2010}{\textbf{HRaidTools: An on-line Suite of Simulation Tools for Heterogeneous RAID systems.} J.L. González, Toni Cortes, Jaime Delgado-Meraz, Ana Piedad-Rubio. 3rd International Conference on Simulation Tools and Techniques (SIMUTools 2010). Malaga, Espa\~na, Marzo 15 -- 19, 2010.}
\cvline{2009}{\textbf{Modelos Actuales Basados en Almacenamiento Adaptivo, Current Approaches Based on Adaptive Storage.} Dalia Rosario Hernández, J.L. González, Jaime Delgado-Meraz. Revista Teczapic. Agosto - Diciembre 2009 Volumen 1 No. 6.}

\subsection{Desarrollos y/o Prototipos}
\cvlistitem{\textbf{SII Tec Valles: Sistema Integral de Información del Instituto Tecnológico de Ciudad Valles}\\Funciones: Programación y mantenimiento.}
\cvlistitem{\textbf{HRaidTools: A simulation Tool for Heterogeneous RAID Systems}\\Funciones: Programación y mantenimiento.}
\cvlistitem{\textbf{Distributed Storage Systems based on Adaptive Redundancy}\\Funciones: Programación y mantenimiento.}
\cvlistitem{\textbf{OES: Online Evaluation System}\\Funciones: Creación, diseño,programación y mantenimiento.}
\cvlistitem{\textbf{ACW: Adaptive Collaborative Work Architecture}\\Funciones: Implementación y mantenimiento}
\cvlistitem{\textbf{Adaptivez}\\Funciones: Webmaster.}

\section{Experiencia Profesional}
\cventry{Enero 2013 a la fecha}{Administrador del Sistema Integral de Información (SII)}{Instituto Tecnológico de Ciudad Valles}{Cd. Valles}{S.L.P}{Realizando actividades de adecuación y actualización de código, cierres de períodos escolares, generación de reportes, respaldos y mantenimientos a la base de datos.}
\cventry{Julio -- Agosto 2011}{Traductor}{Universidad Politécnica de Victoria}{Cd. Victoria, Tamps}{}{Traducción de documentos digitales para la empresa SVAM International.}
\cventry{Mayo -- Julio 2011}{Dise\~nador de contenidos digitales}{Universidad Da Vinci}{Cd. Victoria, Tamps}{}{Manejo de la plataforma Moodle para la publicación de contenidos digitales.}

\section{Conocimientos Informáticos}
\cvcomputer{Sistemas Operativos}{Linux, Windows}{Ofimática}{Certificación Microsoft Office Specialist 2013}
%\cvcomputer{Hardware}{Nivel Intermedio}{Administración Web}{Servidor Apache}
\cvcomputer{Programación Estructurada y OO}{PHP, Java, Visual Basic, Python, C/C++}{Programación Móvil}{J2ME, Android}
\cvcomputer{Programación Web}{PHP, JSP, HTML, XML, CSS, JQuery}{Base de Datos}{PostgreSQL, MySQL, SQLite, Sybase}
\cvcomputer{Maquetación}{\LaTeX{}}{Scripting}{PHP, Bash}
\cvcomputer{SW Educativo}{Moodle, Articulate Studio}{SW de Diseño Gráfico}{Photoshop, GIMP, Inkscape}

\section{Idiomas}
\cvlanguage{Español}{Nativo}{}
\cvlanguage{Inglés}{Alto}{Professional Training for ESL Teachers, TOEFL ITP: 610, TKT Module 2: Band 3}

\section{Premios y Reconocimientos}
\cvline{Junio 2015}{Participación como expositor y autor de la ponencia \textbf{Nenek-EGAD: Esquema de Almacenamiento para la Gestión del Acervo Digitalizado}, presentada en el marco de la \textit{XLII Conferencia Nacional de Ingeniería de la ANFEI}, Ensenada, Baja California.}
\cvline{Octubre 2014}{Impartición del \textbf{Curso -- Taller: Control de Versiones con Git}, realizado en la semana Agentes de Cambio en el marco del XXXIV Aniversario del Instituto, \textit{Instituto Tecnológico de Ciudad Valles}, Cd. Valles, San Luis Potosí. (\textit{5 horas})}
\cvline{Mayo 2014}{Participación como organizador de la \textbf{2a Feria Tecnológica de ISC}, \textit{Instituto Tecnológico de Ciudad Valles}, Cd. Valles, San Luis Potosí.}
\cvline{Febrero 2014}{Impartición del \textbf{Curso -- Taller: Introducción al uso del framework Slim PHP}, \textit{Instituto Tecnológico de Ciudad Valles}, Cd. Valles, San Luis Potosí.}
\cvline{Octubre 2013}{Participación como jurado del \textbf{3er Concurso de Programación}, realizado en la 1er Semana de Ciencia y Tecnología en el marco del XXXIII Aniversario del Instituto, \textit{Instituto Tecnológico de Ciudad Valles}, Cd. Valles, San Luis Potosí}
\cvline{Octubre 2013}{Impartición de la \textbf{Conferencia: Control de Versiones en la nube con Git y Github}, realizado en la 1er Semana de Ciencia y Tecnología en el marco del XXXIII Aniversario del Instituto, \textit{Instituto Tecnológico de Ciudad Valles}, Cd. Valles, San Luis Potosí}
\cvline{Mayo 2013}{Participación como organizador de la \textbf{1a Feria Tecnológica de ISC}, \textit{Instituto Tecnológico de Ciudad Valles}, Cd. Valles, San Luis Potosí.}
\cvline{Febrero 2010}{\textbf{3er lugar de aprovechamiento} de la carrera de Licenciatura en Informática Gen. 2005 -- 2009, \textit{Instituto Tecnológico de Ciudad Valles}, Cd. Valles, San Luis Potosí.}
\cvline{Septiembre 2009}{Participación en el \textbf{XXIV Evento Nacional de Creatividad -- Fase Regional Zona II}, \textit{Instituto Tecnológico de Piedras Negras}, Piedras Negras, Coahuila.}
\cvline{Junio 2009}{\textbf{2o lugar XXIV Evento Nacional de Creatividad -- Fase Local}, \textit{Instituto Tecnológico de Ciudad Valles}, Cd. Valles, San Luis Potosí.}

\section{Formación Complementaria}
\subsection{Enseñanza}
\cventry{Julio 2015}{Curso: Creación de proyectos formativos integradores (PFI) en base a las competencias construidas por las diferentes academias, desde el enfoque socioformativo}{Instituto Tecnológico de Ciudad Valles}{Cd. Valles}{S.L.P}{(\textit{30 horas})}
\cventry{Julio 2014}{Seminario -- Taller: Gestión del curriculum, didáctica y evaluación de competencias desde el enfoque socioformativo}{Instituto Tecnológico de Ciudad Valles}{Cd. Valles}{S.L.P}{(\textit{30 horas})}
\cventry{Enero 2014}{Curso: Planificación y Generación de Estrategias para la consecución de productos del cuerpo académico}{Instituto Tecnológico de Ciudad Valles}{Cd. Valles}{S.L.P}{(\textit{30 horas})}
\cventry{Junio 2013}{Curso: Simulación}{Instituto Tecnológico de Ciudad Valles}{Cd. Valles}{S.L.P}{(\textit{30 horas})}
\cventry{Junio 2013}{Curso: Inducción a la Práctica Docente}{Instituto Tecnológico de Ciudad Valles}{Cd. Valles}{S.L.P}{(\textit{8 horas})}

\subsection{Investigación}
\cventry{Enero 2015}{Curso -- Taller: Cuerpos Académicos - Factor de Integración y Productos del Conocimiento}{Instituto Tecnológico de Ciudad Valles}{Cd. Valles}{S.L.P}{Dirigido a los integrantes del cuerpo académico \emph{Aplicación de Sistemas para el Manejo de Grandes Volúmenes de Datos de Ubicación Geográfica, Científica y Lingüística} (\emph{30 horas})}

\subsection{Tecnologías de la Información}
\cventry{Junio 2015}{Curso -- Taller: Instalación y Capacitación del SII}{Instituto Tecnológico de Ciudad Cuauhtémoc}{Cd. Cuauhtémoc, Chihuahua}{}{Impartido por el Ing. Enrique García Grajeda (\textit{30 horas})}
\cventry{Enero 2015}{Curso -- Taller: Certificación Microsoft Office Specialist (MOS) 2013}{Instituto Tecnológico de Ciudad Valles}{Cd. Valles, S.L.P}{}{Dirigido a docentes y personal de apoyo. (\textit{40 horas})}
\cventry{Junio 2014}{Curso -- Taller: Construcción de nubes privadas para el personal del proyecto Nenek}{Instituto Tecnológico de Ciudad Valles}{Cd. Valles, S.L.P}{}{(\textit{30 horas})}
\cventry{Junio 2014}{Curso de Nubes Privadas y Procesamiento Masivo de Información}{Instituto Tecnológico de Ciudad Valles}{Cd. Valles, S.L.P}{}{Impartido por videoconferencia desde el CINVESTAV-LTI Tamaulipas}
\cventry{Agosto 2010 -- Abril 2011}{Cursos Varios}{Laboratorio de Tecnologías de Información-Tamaulipas del Centro de Investigación y de Estudios Avanzados del Instituto Politécnico Nacional}{Cd. Victoria, Tamps}{}{Cursos de Programación Orientada a Objetos, Matemáticas Discretas, Bases de Datos, Ingeniería de Software, Análisis y Diseño de Algoritmos, Computación Paralela, Cómputo Móvil y Sistemas Distribuidos.}
\cventry{Noviembre 2010}{Taller: Introducción a Blender 3D}{Laboratorio de Tecnologías de Información-Tamaulipas del Centro de Investigación y de Estudios Avanzados del Instituto Politécnico Nacional}{Cd. Victoria, Tamps}{}{Impartido por el Ing. Octavio A. Méndez Sánchez. (\textit{6 horas})}
\cventry{Noviembre 2009}{Curso -- Taller: Espacios Vectoriales en Sistemas de Almacenamiento Distribuido}{Instituto Tecnológico de Ciudad Valles}{Cd. Valles, S.L.P}{}{Impartido por el Dr. Ricardo Marcelín Jiménez.}

\subsection{Inglés}
\cventry{Diciembre 2014}{Teaching Knowledge Test (TKT) Module 2}{Centro de Idiomas Universitario, Universidad Autónoma de San Luis Potosí Zona Huasteca}{Cd. Valles, S.L.P}{}{Certificado expedido por Cambridge English Language Assessment. Puntaje: Band 3}
\cventry{Agosto 2011}{Examen TOEFL ITP}{Universidad Politécnica de Victoria}{Cd. Victoria, Tamps}{}{Puntaje: 610}
\cventry{Junio 2010}{Procesos de Evaluación en la Ense\~nanza de Inglés}{Centro de Idiomas Universitario, Universidad Autónoma de San Luis Potosí Zona Huasteca}{Cd. Valles, S.L.P}{}{Constancia de participación en la Jornada de Evaluación en la Ense\~nanza del idioma Inglés. (\textit{40 horas})}
\cventry{Febrero -- Agosto 2008}{Professional Training for ESL Teachers}{Centro de Idiomas Universitario, Universidad Autónoma de San Luis Potosí Zona Huasteca}{Cd. Valles, S.L.P}{}{Diplomado del idioma Inglés en las áreas de Teorías y Métodos de Ense\~nanza, Redacción, Gramática, Fonética, Vocabulario, Comprensión de Textos, Interpretación, Traducción y Manejo de Grupo. (\textit{160 horas})}
\cventry{Marzo 2008}{Actualización en la Ense\~nanza del idioma Inglés}{Centro de Idiomas Universitario, Universidad Autónoma de San Luis Potosí Zona Huasteca}{Cd. Valles, S.L.P}{}{Seminario de actualización de técnicas de ense\~nanza y uso de multimedios, en el marco de los festejos de la semana cultural, con motivo del XXIV Aniversario del Campus y 85 a\~nos de Autonomía Universitaria. (\textit{320 horas})}
\cventry{Agosto -- Diciembre 2007}{Teacher's Development Seminar II}{Centro de Idiomas Universitario, Universidad Autónoma de San Luis Potosí Zona Huasteca}{Cd. Valles, S.L.P}{}{Seminario del idioma Inglés en las áreas de Vocabulario, Comprensión de Textos, Interpretación, Traducción y Manejo de Grupo. (\textit{320 horas})}
\cventry{Febrero -- Julio 2007}{Teacher's Development Seminar I}{Centro de Idiomas Universitario, Universidad Autónoma de San Luis Potosí Zona Huasteca}{Cd. Valles, S.L.P}{}{Curso Taller del idioma Inglés en las áreas de Conversación, Fonética y Gramática (\textit{240 horas})}
\cventry{Septiembre -- Diciembre 2006}{Teacher's Workshop}{Centro de Idiomas Universitario, Universidad Autónoma de San Luis Potosí Zona Huasteca}{Cd. Valles, S.L.P}{}{Curso -- Taller del idioma Inglés en las áreas de Conversación, Fonética y Gramática. (\textit{70 horas})}
\cventry{2001 -- 2005}{Idioma Inglés Niveles I -- VI}{Centro de Idiomas Universitario, Universidad Autónoma de San Luis Potosí Zona Huasteca}{Cd. Valles, S.L.P}{}{Curso del idioma Inglés (\textit{480 horas})}

\subsection{Otros}
\cventry{Noviembre 2014}{Curso: Redacción de textos académicos}{Universidad Autónoma de San Luis Potosí, Unidad Académica Multidisciplinaria Zona Huasteca}{Cd. Valles, S.L.P}{}{Impartido dentro del marco de la XIV Feria del Libro Cd. Valles UASLP 2014 (\textit{10 horas})}

\section{Referencias}
\cvlistitem{Dr. José Luis González Compéan (\texttt{joseluig@gmail.com})}%\\Depto. de Sistemas y Computación, Instituto Tecnológico de Ciudad Valles.\\(481)38.1.20.44, joseluig@gmail.com
\cvlistitem{Dr. Pedro Luis Sánchez Orellana (\texttt{mcpedrosanchez@gmail.com})}%\\Coordinador de la Maestría en Sistemas Computacionales, División de Estudios de Posgrado e Investigación, Instituto Tecnológico de Ciudad Victoria\\(834)15.3.20.00 ext. 325, pedro\_sanchez@itvictoria.edu.mx
\cvlistitem{Dr. Ricardo Marcelín Jiménez (\texttt{calu@xanum.uam.mx})}%(55)58.04.46.36 ext. 268\\Depto. de Ing. Eléctrica, UAM-Iztapalapa\\

\end{document}
